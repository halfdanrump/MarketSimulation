% Chapter 1

\chapter{Parameter tuning} % Main chapter title

\label{chapter:high_frequency_trading} % For referencing the chapter elsewhere, use \ref{Chapter1} 
$\frac{}{}$
\lhead{Chapter XXX. \emph{XXX}} % This is for the header on each page - perhaps a shortened title
	
%----------------------------------------------------------------------------------------


Chapter \label{chapter:model_calibration} presented the model...XXX

The model has several parameters which must be selected carefully before the simulation can be used for XXX. The parameter tuning has two overall goals, which can be separated as follows. 

First of all, the model must be calibrated such that it mimics the behavior of real markets. Since virtually every aspect of the simulation behavior depends on the values on the various parameters, these must be chosen carefully in order for the simulation to produce realistic behavior. An example of a simulation untuned parameters causing  unrealistic behavior is given in figure \ref{subfig:unrealistic_behavior}. Selecting realistic parameters is by far a simple task. First of all, it requires a way of quantifying the quality of each simulation. The choice of such a quantification is discussed in section \ref{section:simulation_fitness}. Secondly, there might be several different parameter configurations which produce seemingly realistic behavior, but do not correspond to a realistic market setting. An example for this is given in figure \ref{subfig:unrealistic_parameters}, and section \ref{section:filtering_parameters} briefly discusses this point. 
\begin{figure}
\subcaptionbox{Parameters causing unrealistic dynamics\label{subfig:unrealistic_behavior}}
[0.49\linewidth]{\includegraphics[width=0.5\textwidth]{Electron.pdf}}
\subcaptionbox{Unrealistic parameters causing realistic dynamics\label{subfig:unrealistic_parameters}}
[0.49\linewidth]{\includegraphics[width=0.5\textwidth]{Electron.pdf}}
\caption{\textbf{Motivatoin for tuning:} the two}\label{fig:tuning_motivation}
\end{figure}
The second goal of the parameter tuning is to find parameters which promotes certain desirable behaviors. For instance, we might be interested in determining which parameters causes the traded price to stabilize faster after a shock to the fundamental price. Metrics for doing this is discussed in section \ref{section:simulation_fitness}


\begin{enumerate}
\item Fix some of the model parameters in order to reduce the search space for the optimization algorithm.
\item Use an optimization algorithm to find sets of parameters which yield realistic model behavior.
\item From the set of parameter combinations found by the optimization algorithm, remove the parameter combinations which are deemed to be unrealistic.
\end{enumerate}

\subsection{•}
The large number of parameters makes it difficult to use most numerical methods due to the time it takes to evaluate a single set of parameters. Furthermore, the inherent randomness in the model means that the behavior of several simulations with the same set of parameters varies significantly. Therefore, two steps are required.
\begin{enumerate}
\item Reducing the number of parameters that need to be optimized
\item Run the simulation several times with each set of parameters and calculate summarizing statistics of the model behavior
\end{enumerate}
The second point is trivial, and merely requires additional computation, but the first point requires us to consider which parameters can be fixed without making the model less realistic. to do this, we divide the parameters into three categories as follows:
\begin{description}
\item[Fixed across all experiments] These are parameters such as market rules, and simulation settings, 
\item[Fixed for each experiment] These are parameters which 
\item[Optimized for each experiment] These are the 
\end{description}


\subsection{Defining parameter fitness}\label{section:simulation_fitness}
Furthermore, I did not have have access to 

\subsection{Filtering parameters}\label{section:filtering_parameters}
Without available data of the millisecond price movements of a stock in a real market it is difficult to create metrics for measuring the quality of a set of parameters. Instead, we 


As mentioned earlier, it is not enough simply to define a fitness function which assigns high values to parameters causing realistic behavior. In addition, we need some way of 

Instead of 



%http://www.nyse.com/pdfs/NYSE_Euronext_Transactions.pdf

\begin{enumerate}
\item The model must be calibrated such that it mimics the behavior of real markets.
\item The model 
\end{enumerate}

it must be calibrated. The calibration serves two  in order to mimic the dynamics of real financial markets. Without calibration, the simulation will still produce an output, but the chances are high that 







An overview of the model parameters is presented in the end of the thesis, 

\subsubsection{Fixed parameters}
The number of rounds is fixed at $10^5$ for all experiments. The simulation could easily be run for much longer, but due to the increased computational cost 






\section{Optimizing parameters}
