\documentclass[14pt]{beamer}


\setbeamercolor{item projected}{bg=black}
\setbeamertemplate{itemize items}[circle]
\graphicspath{{Figures/}}
\mode<presentation>
{
%  \usetheme{Boadilla}
  \usetheme{CambridgeUS}
  % or ...

  \setbeamercovered{transparent}
  % or whatever (possibly just delete it)
}


%\usepackage[english]{babel}
% or whatever

%\usepackage[latin1]{inputenc}
% or whatever
\usepackage{amsmath,amssymb,latexsym,epsfig,graphicx,psfrag,pstricks, fancybox, ulem}
\usepackage{booktabs}
%\title{Seminar Presentation} % (optional, use only with long paper titles)
\subtitle{Modelling and analysis of a financial market with slow and fast trading agents acting on time-delayed market information}
\title{Master's Thesis}
\author{Halfdan Rump}
\date{February 5 2014} % (optional, should be abbreviation of conference name)
\institute{Waseda GRS-FSE} % (optional, but mostly needed)

\newenvironment{changemargin}[2]{% 
  \begin{list}{}{% 
    \setlength{\topsep}{0pt}% 
    \setlength{\leftmargin}{#1}% 
    \setlength{\rightmargin}{#2}% 
    \setlength{\listparindent}{\parindent}% 
    \setlength{\itemindent}{\parindent}% 
    \setlength{\parsep}{\parskip}% 
  }% 
  \item[]}{\end{list}} 



\begin{document}

\begin{frame}
  \titlepage
\end{frame}






\section{Introduction}
\begin{frame}
\tableofcontents[currentsection]
\end{frame}

\begin{frame}{Background and motivation}
A few fact about modern financial markets:
\begin{itemize}
	\item Humans trade against software algorithms (the machines) 
	\item Humans are slow but complex, whereas algorithms are fast, but (relatively) simple
	\item Fast crashes (flash crashes) has become a problem in recent years
\end{itemize}
\end{frame}

\begin{frame}{Related work}
Models for human/machine system must be developed. Previous work:
\begin{description}
	\item[Analysis of market data] Works analyzing real market data for flash crashes and
	\item[Models of markets] Works that divide agents into two groups: \textbf{slow} and \textbf{fast} traders
\end{description}
All discovered works in the field are recent (published 2013, or yet unpublished).
\end{frame}

\begin{frame}{Key points of proposed model}
\begin{description}
\item[Delayed market information] All information exchanged between agents and the market is delayed
\item[Agents with arbitrary time delays] Agents are not just \textit{fast} or \textit{slow}, but have \textbf{arbitrary} delays
\item[Full-fledged MAS model] Agents with various strategies and different delays.
\end{description}
\end{frame}

\begin{frame}{Research goal}
\begin{block}{Market stability and agent speed}
Investigate how the behavior (e.g. stable, crash, etc.) of a simulated financial market changes when the latency of the traders change
\end{block}
Very open research, but the first steps in a new field must necessarily be somewhat exploratory.
\end{frame}


%\begin{frame}{Methodology}
%
%\begin{itemize}
%content...
%\end{itemize}
%\end{frame}






\section{Model}
\begin{frame}
\tableofcontents[currentsection]
\end{frame}

\begin{frame}{Model components}
\begin{itemize}
\item Market and order book
\item Stock
\item Agents
\item Messages
\end{itemize}
\end{frame}

\begin{frame}{Market model}
\begin{center}
\includegraphics[width=0.7\textwidth]{graph.png}
\end{center}
\end{frame}


\begin{frame}{Market}
\begin{block}{Auction type}
The market uses a continuous double auction. Prices are always executed at the prices of the best standing market orders.
\end{block}
\end{frame}


\begin{frame}{Order book}
\begin{table}
\centering
\begin{tabular}{l|c|r}
Sell orders & Price & Buy orders\\
\midrule
22 & 9994 &\\
26 & 9993 &\\
13 & 9992 &\\
10 & 9991 &\\ 
{}&{}&{}\\
& 9988& 12\\
& 9987& 10\\
& 9986& 16\\
& 9985& 25\\
\end{tabular}
\end{table}
\end{frame}

\begin{frame}{Messages}
\begin{itemize}
\item Market information
\item Orders
\item Receipts
\item Cancellations
\end{itemize}
\textit{All messages have a non-zero travel time}
\end{frame}

\begin{frame}{Stock}
A single stock is traded at the market.
\begin{description}
\item[Fundamental price] The ``\textit{true}'' value of the stock
\item[Traded price] The price at which the stock is currently traded
\end{description}
\end{frame}


\begin{frame}{Agents}
\begin{itemize}
\item Slow traders (ST)
\item Fast traders (AKA High Frequency Traders, HFT)
\begin{itemize}
\item Market makers (MM)
\item Simple chartists (SC)
\end{itemize}
\end{itemize}
\end{frame}



\begin{frame}{Slow traders}
\begin{block}{Slow traders model human traders}
They know the \textbf{true} value of the fundamental, \textit{but with a large delay}
\end{block}
The slow traders submit orders in order to \textit{move the trade price towards the true price}.
\end{frame}


\begin{frame}{Simple chartists}
\begin{block}{The chartists use a simple moving average strategy}
They calculate the moving from the delayed best buy and sell prices. 
\end{block}
The chartist detects a trend if the moving average calculated over $H_c$ rounds differs more than $\gamma_c$ ticks from the currently traded price.
\end{frame}

\begin{frame}{Chartist example 1}
\begin{center}
\includegraphics[width=0.7\textwidth]{chartist/b.png}
\end{center}
\end{frame}

\begin{frame}{Chartist example 2}
\begin{center}
\includegraphics[width=0.7\textwidth]{chartist/h.png}
\end{center}
\end{frame}



\begin{frame}{Market makers}
\begin{block}{Market makers keep constant buy and sell orders}
The market maker tried to follow the best buy/sell prices to stay competitive.
\end{block}
The market maker has a minimum-spread parameter, $\theta$.
\end{frame}

\begin{frame}{Market maker case 1}
\begin{center}
\includegraphics[width=0.7\textwidth]{marketmaker/a.png}
\end{center}
\end{frame}

\begin{frame}{Market maker case 2}
\begin{center}
\includegraphics[width=0.7\textwidth]{marketmaker/b.png}
\end{center}
\end{frame}

\begin{frame}{Market maker case 3}
\begin{center}
\includegraphics[width=0.7\textwidth]{marketmaker/c.png}
\end{center}
\end{frame}



\begin{frame}{Important model parameters}
The model has many parameters. The most important ones are:
\begin{itemize}
\item The average latency of chartists and of market makers, $\lambda$
\item The number of fast traders
\end{itemize}
\end{frame}

\section{Experiments}
\begin{frame}
\tableofcontents[currentsection]
\end{frame}

\begin{frame}{Simulating bad news}
\begin{block}{Shock to the fundamental}
How does the market react when the true price of the stock suddenly drops?
\end{block}
\end{frame}

\begin{frame}
\begin{center}
\includegraphics[width=0.7\textwidth]{market_cases/a_stable_within_margin.png}
\end{center}
\end{frame}

\begin{frame}{Exploring model behavior}
\begin{block}{We want to determine how the model behavior changes with the parameters}
A genetic algorithm was used to search the parameter space. Four fitness measures were defined in order to quantify the model behavior.
\end{block}
\end{frame}

\begin{frame}{Model fitness}
\begin{itemize}
\item Overshoot
\item Price flickering (standard deviation of trade prices)
\item Response time (time to reach fundamental price after shock)
\item Time to become stable (the traded price must stay within a certain range of the true price)
\end{itemize}
\end{frame}

\begin{frame}
\begin{center}
\includegraphics[width=0.7\textwidth]{market_cases/a_stable_within_margin.png}
\end{center}
\end{frame}

\begin{frame}{Search for stable markets}
\begin{block}{We want to see what the speed of the agents does to the market stability}
The genetic algorithm was instructed to minimize overshoot, price flickering, response time and time to become stable.
\end{block}
\end{frame}


	
\section{Results}
\begin{frame}
\tableofcontents[currentsection]
\end{frame}

\subsection{Evolution of fitness and parameters}
\begin{frame}
\tableofcontents[currentsection]
\end{frame}

\begin{frame}{Overshoot}
The genetic algorithm found markets with a small overshoot:
\begin{center}
\includegraphics[width=0.7\textwidth]{evolution/overshoot.png}
\end{center}
\end{frame}

\begin{frame}{Time to become stable}
..it found markets that quickly staying within the stability margin:
\begin{center}
\includegraphics[width=0.7\textwidth]{evolution/round_stable.png}
\end{center}
\end{frame}

\begin{frame}{Price flickering}
...it found markets with little price flickering:
\begin{center}
\includegraphics[width=0.7\textwidth]{evolution/stdev.png}
\end{center}
\end{frame}

\begin{frame}{Response time}
...but it was not able to find markets that were fast at the same time
\begin{center}
\includegraphics[width=0.7\textwidth]{evolution/time_to_reach_new_fundamental.png}
\end{center}
\end{frame}

\begin{frame}{Interpretation of evolution results}
\begin{block}{Speed/stability trade-off}
The evolution of the fitness and parameters illustrates a trade-off between maket stability and the response time of the market: Stable market are also slower.
\end{block}
What parameters cause this behavior?
\end{frame}


\begin{frame}{Number of market makers}
The GA selects towards more market makers:
\begin{center}
\includegraphics[width=0.7\textwidth]{evolution/nAgents.png}
\end{center}
\end{frame}

\begin{frame}{Latency of agents}
...and towards slower agents.
\begin{center}
\includegraphics[width=0.7\textwidth]{evolution/latpars_mu.png}
\end{center}
\end{frame}

\subsection{Typical markets}
\begin{frame}
\tableofcontents[currentsubsection]
\end{frame}


\begin{frame}{Market with nice behavior}
\begin{center}
\includegraphics[width=0.7\textwidth]{market_cases/a_stable_within_margin.png}
\end{center}
\end{frame}

\begin{frame}{Large price flickers}
\begin{center}
\includegraphics[width=0.7\textwidth]{market_cases/b_flicker_but_mostly_within_margin.png}
\end{center}
\end{frame}

\begin{frame}{Never stable}
\begin{center}
\includegraphics[width=0.7\textwidth]{market_cases/c_flickering_on_margin.png}
\end{center}
\end{frame}

\begin{frame}{Overvaluation}
\begin{center}
\includegraphics[width=0.7\textwidth]{market_cases/d_never_reach_fundamental.png}
\end{center}
\end{frame}

\begin{frame}{Undervaluation}
\begin{center}
\includegraphics[width=0.7\textwidth]{market_cases/e_stable_under_fundamental.png}
\end{center}
\end{frame}

\begin{frame}{Market crash}
\begin{center}
\includegraphics[width=0.7\textwidth]{market_cases/f_crash.png}
\end{center}
\end{frame}

\begin{frame}{Supercrash}
\begin{center}
\includegraphics[width=0.7\textwidth]{market_cases/shouganai.png}
\end{center}
\end{frame}


\subsection{Fitness/parameter correlations}
\begin{frame}
\tableofcontents[currentsubsection]
\end{frame}


\begin{frame}{Parameters and model behavior}
\textit{How does the speed of the agents affect the market behavior?}
\end{frame}

\begin{frame}{Chartist latency}
Faster chartists cause the market to have a larger overshoot:
\begin{center}
\includegraphics[width=0.7\textwidth]{scatter/sc_latency_mu__vs__overshoot(mean)_scatter.png}
\end{center}
\end{frame}

\begin{frame}{Number of market makers}
The number of market makers has a great deal to say:
\begin{center}
\includegraphics[width=0.7\textwidth]{scatter/ssmm_nAgents__vs__overshoot(mean)_scatter.png}
\end{center}
\end{frame}

\begin{frame}{Number of chartists}
...as has the number of chartists:
\begin{center}
\includegraphics[width=0.7\textwidth]{scatter/sc_nAgents__vs__overshoot(mean)_scatter.png}
\end{center}
\end{frame}

\subsection{Parameter ratios}
\begin{frame}
\tableofcontents[currentsubsection]
\end{frame}

\begin{frame}{Latency ratio}
\textit{What happens when the chartists are faster than the market makers, or the other way around?}
\end{frame}

\begin{frame}{Overshoot}
\begin{center}
\includegraphics[width=0.7\textwidth]{ratio/overshoot.png}
\end{center}
\end{frame}


\begin{frame}{Log-overshoot}
\begin{center}
\includegraphics[width=0.7\textwidth]{ratio/overshoot_log.png}
\end{center}
\end{frame}


\begin{frame}{Market response time}
\begin{center}
\includegraphics[width=0.7\textwidth]{ratio/time_to_reach_new_fundamental.png}
\end{center}
\end{frame}

\begin{frame}{Summary of results}
\begin{itemize}
\item Market makers have a stabilizing effect of the market by reducing price movements
\item Chartists increase price movements
\item The influence was larger for faster agents
\item The market will only crash if \textit{both} chartists \textit{and} market makers are present.
\item The market was more likely to crash if the chartists were much faster than the market makers
\end{itemize}
\end{frame}

\begin{frame}{Log-overshoot}
\begin{center}
\includegraphics[width=0.7\textwidth]{ratio/overshoot_log.png}
\end{center}
\end{frame}


\section{Conclusion}
\begin{frame}
\tableofcontents[currentsection]
\end{frame}

\begin{frame}{Conclusions}
\begin{description}
\item[Fast trader pros and cons] Fast agents both provided benefits to the market (faster response, lower price flickering), and caused some dangers (crashes, misvaluation)
\item[Stability/speed trade-off] The market was found to have a trade-off between speed and stability
\item[Relative agent latencies] The results illustrated the importance of allowing different agents to have different latencies.
\end{description}
\end{frame}

%We have more control over what simplifications we make than we have over events that we never observed.





%frame: black swans: some events are fundamentally unpredictable. Theories based on a few observations can never hope to predict these, as (only training data, no test data)




%frame: what I hope to be doing for the next two years. Artificial markets and HFT




%frame[advantages] - we can control conditions - we can repeat experiments - we can observe general trends and patterns

%frame[problems] - we need to simplify a lot - how do we simplify? - difficult to know if simplifications are OK or not - difficult to validate results obtained through simulation

%frame - emphasis on bottom up tinkering: create environment first, then play around - some top down approach is required when designing environment




\begin{thebibliography}{Bib}

\bibitem{mcgowan}
  Michael, J. McGowan,
  \emph{The Rise of Computerized High Frequency Trading: Use and Controversy}.
  Duke Law \& Technology Review,
  2012.
 \bibitem{izumi}
  K. Izumi, F. Toriumi, H. Matsui
  \emph{Evaluation of automated strategies using an artificial market}.
  Neurocomputing,
  2009.
  \bibitem{cincotti}
  S. Cincotti, S.M. Focardi, L. Ponta, M. Raberto, E. Scalas
  \emph{The waiting-time distribution of trading activity in a double auction artificial financial market}. Unpublished, 2011
  \bibitem{luca}
  M. De Luca, C. Szostek, J. Cartlidge, D. Cliff
  \emph{Studies of interactions between human traders and algorithmic trading systems}.
  Commissioned as part of the UK {Government's} Foresight Project, The Future of Computer Trading in Financial Markets--Foresight Driver Review--DR 13, 2011
  \bibitem{johnson}
  N. Johnson, G. Zhao, E. Hunsader, J. Meng, A. Ravindar, S. Carran, B. Tivnan
  \emph{Financial black swans driven by ultrafast machine ecology}.
  Submitted, 2012.
   

\end{thebibliography}

\end{document}


