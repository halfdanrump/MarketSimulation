\chapter{Discussion}\label{chapter:discussion}

\section{Trader speed and market overshoot}

\section{Trader speed and price stability}
Trade-off between responsiveness and stability.

Refer to the section of the best individuals. 
Some simulations were fast and unstable, while others were slow and stable. Although there were models in between, no stable model managed to be as fast as another unstable model. and no fast model managed to be as stable as a stable one. 

The trade-off between these two types of behavior was shown to be 

Whether such ideal market behavior is possible within the model is doubtful. 



As was shown in figure \ref{figure:d10_fitness_correlation}, \timetoreachnewfundamental{} is negatively correlated with both \stdev{} and \overshoot{}. This is another factor that points towards the existence between responsiveness and stability. The negative correlation means that \timetoreachnewfundamental will be high when \stdev and \overshoot are low and vice versa. 
XXX: insert figure of D11 fitness correlation

\section{Ratio of agents}
In chapter \ref{chapter:experiments_and_results} it was shown that a correlation exists between the number of high frequency traders and some of the fitness measures. 

It is possible that this evolution of the parameters is caused by the model being unable to be both fast and stable, but that there exists a trade-off between the two. Even though such a trade-off is not inherent in the model, in may happen to be inevitable when performing a parameter search, simply because the set of parameters causing the model to be both responsive and stable are so few that they are simply too difficult to find. Indeed, as is shown in section \ref{section:hall_of_fame}, no parameters caused the market to be both 


\section{Number of trades}
The market maker strategy is designed such that a market maker can only have a single buy order and a single sell order in place at the market at the same time. When an order submitted by market maker $i$ is filled, it takes \ssmmlatencyagent{i} rounds before the agent is notified of the result. When the market maker receives trade information, it thinks for \ssmmthinkagent{i} rounds, and then submits a new order to the market, which arrives after another \ssmmlatencyagent{i} rounds. Recall that market maker $i$ submits orders with a fixed volume, \ssmmordervolume{i}. The maximum volume that the market maker can move during $n$ rounds is therefore inversely proportional to its latency to the market:
\[\text{Maximum volume} \propto 2\ssmmlatencyagent{i} + \ssmmthinkagent{i}\]

has to wait until it receives the new market information until it can make a decision on how to trade. Hence, the number of orders that 

\section{Strategy crowding}

\section{Between data set inconsistencies}

\section{Model assumptions}

\section{Model limitations}
The most obvious limitation imposed by the model is the small number of different agent strategies.

Another possible short-coming is the simplicity of the slow trader strategy. 